\documentclass[conference]{IEEEtran}
\IEEEoverridecommandlockouts
% The preceding line is only needed to identify funding in the first footnote. If that is unneeded, please comment it out.
%\usepackage{cite}
\usepackage{amsmath,amssymb,amsfonts}
\usepackage{algorithmic}
\usepackage{graphicx}
\usepackage{textcomp}
\usepackage{xcolor}
\def\BibTeX{{\rm B\kern-.05em{\sc i\kern-.025em b}\kern-.08em
    T\kern-.1667em\lower.7ex\hbox{E}\kern-.125emX}}

\usepackage{hyperref}
\usepackage{biblatex} 
\addbibresource{references.bib}

\begin{document}

\title{Conference Paper Title*
}

\author{\IEEEauthorblockN{Yvette Espinoza}
\IEEEauthorblockA{\textit{dept. name of organization (of Aff.)} \\
\textit{name of organization (of Aff.)}\\
Los Angeles, CA \\
yespinoz@purdue.edu}
\and
\IEEEauthorblockN{2\textsuperscript{nd} Given Name Surname}
\IEEEauthorblockA{\textit{dept. name of organization (of Aff.)} \\
\textit{name of organization (of Aff.)}\\
City, Country \\
email address or ORCID}
\and
\IEEEauthorblockN{3\textsuperscript{rd} Given Name Surname}
\IEEEauthorblockA{\textit{dept. name of organization (of Aff.)} \\
\textit{name of organization (of Aff.)}\\
City, Country \\
email address or ORCID}
}

\maketitle

\begin{abstract}
TBD
\end{abstract}

\begin{IEEEkeywords}
component, formatting, style, styling, insert
\end{IEEEkeywords}

\section{Introduction}

Early research in activity detection aimed at finding differences in signal characteristics of the collected accelerometer data, and determining the ideal sensor placement. 
In \cite{2011_Sensor_Positioning}, activities were grouped together into four categories by levels of physical activity, and sensors were placed throughout the body to determine the best location for each of the categories.
Their results found each category performed best with a different sensor location. A sensor placed on the waist was best at detecting low level activities like eating, while a sensor on the chest or wrist was best at medium level activities like housework.

While different sensor placement based on an activity would be ideal for performance, it would not be practical for widespread use.
An accelerometer by itself is not enough to determine an activity, context would be needed, and different sensors on the body are an inconvenience for users. 
Human activity recognition is widely used in healthcare applications, with the elderly being a large part of the user demographics \cite{2018_Robust_Activity}, to suit their needs the data collection would require an unobtrusive setup.
To address the inconvenience of full body sensors and the need for context, \cite{2012_WristSense} used a wrist-worn device equipped with an accelerometer and camera to recognize daily activities. 
The camera provided context for the activities, and the accelerometer provided characteristics of the body movement associated with different activities, both of which were used to train a model to predict the activity.

The advancement in smart devices, like smartwatches and conversational assistants, allow for more sensors in a user friendly device.
Conversational assistants, like a Google Home, and smartwatches are used to train a model to recognize activities of daily living \cite{2021_Ok_Google} \cite{2022_Leveraging_sound}.

% YVETTE - talk about the machine learning advances --> smart devies allow for more data collection, which allow for better ML models and classifiers %


\section{Section}

\subsection{Subsection}

\section*{Acknowledgment}

Paper we are implementing \cite{2022_Leveraging_sound}.

\nocite{*}
\printbibliography

\end{document}
